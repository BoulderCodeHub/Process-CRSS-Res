\PassOptionsToPackage{unicode=true}{hyperref} % options for packages loaded elsewhere
\PassOptionsToPackage{hyphens}{url}
%
\documentclass[
]{article}
\usepackage{lmodern}
\usepackage{amssymb,amsmath}
\usepackage{ifxetex,ifluatex}
\ifnum 0\ifxetex 1\fi\ifluatex 1\fi=0 % if pdftex
  \usepackage[T1]{fontenc}
  \usepackage[utf8]{inputenc}
  \usepackage{textcomp} % provides euro and other symbols
\else % if luatex or xelatex
  \usepackage{unicode-math}
  \defaultfontfeatures{Scale=MatchLowercase}
  \defaultfontfeatures[\rmfamily]{Ligatures=TeX,Scale=1}
\fi
% use upquote if available, for straight quotes in verbatim environments
\IfFileExists{upquote.sty}{\usepackage{upquote}}{}
\IfFileExists{microtype.sty}{% use microtype if available
  \usepackage[]{microtype}
  \UseMicrotypeSet[protrusion]{basicmath} % disable protrusion for tt fonts
}{}
\makeatletter
\@ifundefined{KOMAClassName}{% if non-KOMA class
  \IfFileExists{parskip.sty}{%
    \usepackage{parskip}
  }{% else
    \setlength{\parindent}{0pt}
    \setlength{\parskip}{6pt plus 2pt minus 1pt}}
}{% if KOMA class
  \KOMAoptions{parskip=half}}
\makeatother
\usepackage{xcolor}
\IfFileExists{xurl.sty}{\usepackage{xurl}}{} % add URL line breaks if available
\IfFileExists{bookmark.sty}{\usepackage{bookmark}}{\usepackage{hyperref}}
\hypersetup{
  pdftitle={April 2020 CRSS Updates},
  pdfborder={0 0 0},
  breaklinks=true}
\urlstyle{same}  % don't use monospace font for urls
\usepackage[margin=1in]{geometry}
\usepackage{longtable,booktabs}
% Allow footnotes in longtable head/foot
\IfFileExists{footnotehyper.sty}{\usepackage{footnotehyper}}{\usepackage{footnote}}
\makesavenoteenv{longtable}
\usepackage{graphicx,grffile}
\makeatletter
\def\maxwidth{\ifdim\Gin@nat@width>\linewidth\linewidth\else\Gin@nat@width\fi}
\def\maxheight{\ifdim\Gin@nat@height>\textheight\textheight\else\Gin@nat@height\fi}
\makeatother
% Scale images if necessary, so that they will not overflow the page
% margins by default, and it is still possible to overwrite the defaults
% using explicit options in \includegraphics[width, height, ...]{}
\setkeys{Gin}{width=\maxwidth,height=\maxheight,keepaspectratio}
\setlength{\emergencystretch}{3em}  % prevent overfull lines
\providecommand{\tightlist}{%
  \setlength{\itemsep}{0pt}\setlength{\parskip}{0pt}}
\setcounter{secnumdepth}{-2}
% Redefines (sub)paragraphs to behave more like sections
\ifx\paragraph\undefined\else
  \let\oldparagraph\paragraph
  \renewcommand{\paragraph}[1]{\oldparagraph{#1}\mbox{}}
\fi
\ifx\subparagraph\undefined\else
  \let\oldsubparagraph\subparagraph
  \renewcommand{\subparagraph}[1]{\oldsubparagraph{#1}\mbox{}}
\fi

% set default figure placement to htbp
\makeatletter
\def\fps@figure{htbp}
\makeatother

\usepackage[none]{hyphenat}

\title{April 2020 CRSS Updates}
\author{}
\date{\vspace{-2.5em}}

\begin{document}
\maketitle

This document includes all of the changes made to CRSS between the
``February 2020'' and the ``April 2020'' packages. Files for the model
and ruleset for these two packages are:

\begin{longtable}[]{@{}lll@{}}
\toprule
& April 2020 & February 2020\tabularnewline
\midrule
\endhead
RiverWare & RW v8.0.3 & RW v7.5.2\tabularnewline
Model & CRSS.V4.5.0.2021.Apr2020.mld &
CRSS.V4.4.1.2021.Feb2020.mdl\tabularnewline
Ruleset & CRSS.Baseline.2027{[}IGDCP/NA{]}.v4.4.1.rls &
CRSS.Baseline.2027{[}IGDCP/NA{]}.v4.4.0.rls\tabularnewline
\bottomrule
\end{longtable}

\hypertarget{changes-that-affect-results}{%
\subsection{Changes That Affect
Results}\label{changes-that-affect-results}}

\begin{itemize}
\tightlist
\item
  Mead's live capacity storage
  (\texttt{UBRuleCurveData.ReservoirData{[}"Mead",\ "liveCapacityStorage"{]}})
  was set to 27,620,000 acre-ft to match the relatively recent updates
  to Mead's area-capacity tables. (Mead's elevation-volume tables were
  previously updated, but this slot was missed in that update.) Updating
  the Mead live capacity value affects computations in the Mead Flood
  Control rule. Overall, this affect 30.10 \% of months at Mead, with an
  average absolute difference of 0.46 ft. When Mead's elevations change,
  this can have upstream affects on Powell. 11.96 \% of months are
  affected at Powell with an average absolute difference of 0.14 ft.
\item
  \raggedright Navajo's ``liveCapacityStorage'' and
  ``inactiveCapacityStorage'' were updated on
  \texttt{UBRuleCurveData.ReservoirData{[}{]}}. Changing Navajo's live
  capacity affects the computation of the Powell inflow ``forecast''
  (within CRSS -
  \texttt{PowellForecastData.Reg\ Inflow\ with\ Error{[}{]}}), which in
  turn affects all of the EOWY forecasts through changing the total
  volume computed in \texttt{UBEffectiveStorage()}. Of course changing
  these EOWY storage forecasts at Powell can affect Powell and Mead
  elevations. Overall, 52.90 and 55.00 \% of months are affected at
  Powell and Mead, respectively. There is an average absolute difference
  of 0.14 and 0.16 feet, respectively.
\item
  There was a typo in the
  ``DIT\_CRSS\_UCRC2007shortageEIS2010\_FINALv4.5.xlsm'' Demand Input
  Tool in the \texttt{YumaOperations.WelltonMohawkBypassFlows} water
  user. The typo caused this user to have a decreasing demand from
  115,992 acre-ft in 2000 to 115,922 acre-ft in 2060; however, we intend
  to model this as a static value (115,922 acre-ft) computed from the
  1990-2018 historical average. The typo was fixed, and this causes very
  minor differences in results as the change in the demand was small (70
  acre-ft at most). This file is now saved as v4.6.
\item
  Parameters in Havasu's power method were modified to match those used
  in the 24-Month Study and MTOM. This does affect energy projections,
  but it does not affect reservoir releases or elevations.
\end{itemize}

\hypertarget{changes-that-do-not-affect-results}{%
\subsection{Changes That Do Not Affect
Results}\label{changes-that-do-not-affect-results}}

\hypertarget{model-and-global-functions}{%
\subsubsection{Model and Global
Functions}\label{model-and-global-functions}}

\begin{itemize}
\tightlist
\item
  The DMIs used to import ``VIC'' demands (the agricultural and M\&I
  demands that are scaled based on Global Circulation Model projections
  of changes in ET) were converted from control file executable style
  DMIs to trace based DMIs. All other DMIs were previously converted to
  trace based DMIs which removes the historical dependency on Perl.
\item
  All MRM configurations are now set to ``Stop all distributed MRM runs
  on error''. This new RiverWare feature ensures that if one trace has
  an error, you will not waste time completing all other traces when you
  are using Distributed MRM.
\item
  The ``Update MRM rulesets'' RiverWare script was added to the model
  file. This helps us change the ruleset that every MRM configuration
  uses at once.
\end{itemize}

\hypertarget{ruleset}{%
\subsubsection{Ruleset}\label{ruleset}}

In both the IG\_DCP and NA rulesets:

\begin{itemize}
\tightlist
\item
  Moved rule `Set ICS Put and Take Dates' to the lowest priority rule.
  Rule is now part of a new policy group `Set Operational Dates'. This
  makes this rule easier to find and indicates that it should be the
  very first thing to happen in the model run.
\end{itemize}

\hypertarget{other-files}{%
\subsubsection{Other Files}\label{other-files}}

\begin{itemize}
\tightlist
\item
  \raggedright A new event (``RWcheck'') has been added to the
  RiverSMART study file (CRSS\_StudyManager\_Apr2020\_MTOM.rsm). This
  event will eventually perform many automated checks on CRSS output to
  check results for errors. It is still in development and testing and
  we will provide more information on it once it is more complete.
\end{itemize}

\end{document}
